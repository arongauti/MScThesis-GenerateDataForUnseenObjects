\chapter{Results}
\textit{\ifdraft{In this section you discuss any issues that came up while developing
the system.  If you found something particularly interesting,
difficult, or an important learning experience, put it here.  This is
also a good place to put additional figures and data.}}

\section{Using a robot to create image dataset}\label{resrobotcontrol}

\subsection*{Tests}
Robot manipulator performance was measured by making four tests and taking time, iterations and interventions. The performance can been seen in \textit{Table \ref{tab:testonrobot}}.

\begin{table}[h]
\resizebox{\textwidth}{!}{%
\begin{tabular}{clccccccc}
\hline
\textit{Test\#} &
  \textit{Item} &
  \textit{\begin{tabular}[c]{@{}c@{}}Start pos\\ {[}x, y, z{]}\end{tabular}} &
  \textit{Iterations} &
  \textit{\begin{tabular}[c]{@{}c@{}}Operator\\ intervention\end{tabular}} &
  \textit{\begin{tabular}[c]{@{}c@{}}Time \\ {[}sec{]}\end{tabular}} &
  \textit{\begin{tabular}[c]{@{}c@{}}Did it \\ finish?\end{tabular}} &
  \textit{\begin{tabular}[c]{@{}c@{}}Movement \\ time {[}sec{]}\end{tabular}} &
  \textit{\begin{tabular}[c]{@{}c@{}}Intervention\\ vs.  Iterations\end{tabular}} \\ \hline
\multicolumn{1}{c|}{1} &
  \begin{tabular}[c]{@{}l@{}}Nivea \\ Cleansing Milk\end{tabular} &
  \begin{tabular}[c]{@{}c@{}}{[}0.336, \\ 0.045, \\ 0.097{]}\end{tabular} &
  100 &
  2 &
  1277.2 &
  Yes &
  12.77 &
  2\% \\
\multicolumn{1}{c|}{2} &
  \begin{tabular}[c]{@{}l@{}}Alberto \\ Balsam coconut\end{tabular} &
  \begin{tabular}[c]{@{}c@{}}{[}0.343, \\ 0.043, \\ 0.107{]}\end{tabular} &
  100 &
  0 &
  1254.2 &
  Yes &
  12.54 &
  0\% \\
\multicolumn{1}{c|}{3} &
  \begin{tabular}[c]{@{}l@{}}Nivea \\ Cleansing Milk\end{tabular} &
  \begin{tabular}[c]{@{}c@{}}{[}0.340 , \\ 0.044, \\ 0.098{]}\end{tabular} &
  300 &
  6 &
  3799.1 &
  Yes &
  12.66 &
  2\%  \\
\multicolumn{1}{c|}{4} &
  \begin{tabular}[c]{@{}l@{}}Alberto \\ Balsam coconut\end{tabular} &
  \begin{tabular}[c]{@{}c@{}}{[}0.333 , \\ -0.040, \\ 0.118{]}\end{tabular} &
  300 &
  1 &
   3774.4&
  Yes &
   12.58 &
  0.33\% \\ \hline
\multicolumn{7}{r}{\textbf{Average:}} &
  12.64 &
  1.08\% \\ \hline
\end{tabular}%
}
\caption{Test made on the robot and code performance}
\label{tab:testonrobot}
\end{table}

%%%%%%%%%%%%%%%%%%%%%%%%%%%%%%%%%%%%%%%%%%%%%%%%%%%%%%%%%%%%%%%%%%%%%
\section{Automatic labelling}\label{rescamera}

\subsection*{Before vs. after}
\begin{figure}[h]
    \centering
    % include first image
    \subfloat[Before]{\includegraphics[width=0.45\textwidth]{graphics/9before.png}}
    \hfill
    \subfloat[After]{\includegraphics[width=0.45\textwidth]{graphics/9after.png}}
    \hfill
    \subfloat[Filled]{\includegraphics[width=0.45\textwidth]{graphics/9filled.png}}
    \hfill
    \subfloat[Masked]{\includegraphics[width=0.45\textwidth]{graphics/9masked.png}}
    \caption{Image Difference with OpenCV and Python}
    \label{figure: imagework}
\end{figure}

As can been in the \textit{Figure \ref{figure: imagework}} this would not work so another method would be needed to find the right bounding box automatically. That method is described in methods in \textit{Section \ref{sec:difference}}.

\subsection*{Empty bin vs. Object in bin}
\begin{figure}[h]
    \centering
    % include first image
    \subfloat[Alberto Balsam]{\includegraphics[width=0.45\textwidth]{graphics/results/albertobalsam100_51.png}}
    \hfill
    \subfloat[Nivea Cleansing Milk]{\includegraphics[width=0.45\textwidth]{graphics/results/niveacleansingmilk100_8.png}}
    \hfill
    \subfloat[Nivea Elastic]{\includegraphics[width=0.45\textwidth]{graphics/results/niveaelastic100_0002box.png}}
    \hfill
    \subfloat[Nivea texture]{\includegraphics[width=0.45\textwidth]{graphics/results/niveatexture70_27.png}}
    \caption{The results from the difference.py, which shows the bounding box of four products}
    \label{figure: labelling}
\end{figure}

% Please add the following required packages to your document preamble:
% \usepackage{graphicx}
% Please add the following required packages to your document preamble:
% \usepackage{graphicx}
\begin{table}[h]
\resizebox{\textwidth}{!}{%
\begin{tabular}{clccc}
\multicolumn{1}{l|}{\textit{Test \#}} &
  \textit{Item} &
  \multicolumn{1}{l}{\textit{Images}} &
  \multicolumn{1}{l}{\textit{Time {[}s{]}}} &
  \multicolumn{1}{l}{\textit{Time per image {[}s{]}}} \\ \hline
\multicolumn{1}{c|}{1} & \begin{tabular}[c]{@{}l@{}}Nivea\\ cleansing milk\end{tabular}    & 300 & 255.99 & 0.85 \\
\multicolumn{1}{c|}{2} & \begin{tabular}[c]{@{}l@{}}Nivea\\ cleansing milk\end{tabular}    & 102 & 92.68  & 0.91 \\
\multicolumn{1}{c|}{3} & \begin{tabular}[c]{@{}l@{}}Nivea \\ elastic\end{tabular}          & 72  & 56.31  & 0.78 \\
\multicolumn{1}{c|}{4} & \begin{tabular}[c]{@{}l@{}}Alberto \\ Balsam cocunut\end{tabular} & 102 & 88.30  & 0.87 \\ \hline
\multicolumn{4}{r}{Average:}                                                                              & 0.85
\end{tabular}%
}
\caption{Measured time when using the difference.py}
\label{tab:timediff}
\end{table}

\clearpage

%%%%%%%%%%%%%%%%%%%%%%%%%%%%%%%%%%%%%%%%%%%%%%%%%%%%%%%%%%%%%%%%%%%%%
\section{Creating a new model for object detection}\label{reslabelled}
\begin{figure}[h]
    \centering
    % include first image
    \subfloat[Using YOLOv4]{\includegraphics[width=0.33\textwidth]{graphics/results/beforetraining.png}}
    \hfill
    \subfloat[Using YOLOv4]{\includegraphics[width=0.33\textwidth]{graphics/results/beforetraining1.png}}
    \hfill
    \subfloat[Using YOLOv4]{\includegraphics[width=0.33\textwidth]{graphics/results/beforetraining2.png}}
    \hfill
    \subfloat[Using trained YOLOv4]{\includegraphics[width=0.33\textwidth]{graphics/results/aftertraining.png}}
    \hfill
    \subfloat[Using trained YOLOv4]{\includegraphics[width=0.33\textwidth]{graphics/results/aftertraining1.png}}
    \hfill
    \subfloat[Using trained YOLOv4]{\includegraphics[width=0.33\textwidth]{graphics/results/aftertraining2.png}}
    \caption{Using YOLOv4 and the trained YOLOv4 model, trained on images from the robot}
    \label{figure: beforeaftertraining}
\end{figure}

% GPU: GeForce RTX 2080 Ti

% Net fra 25.04.21
% yolo-obj_new_1000.weights average for conf_thresh = 0.25, TP = 91, FP = 31, FN = 0, average IoU = 51.78 %
% yolo-obj_new_2000.weights average for conf_thresh = 0.25, TP = 91, FP = 0, FN = 0, average IoU = 84.03 %
% yolo-obj_new_3000.weights average for conf_thresh = 0.25, TP = 91, FP = 0, FN = 0, average IoU = 90.02 % 
% yolo-obj_new_4000.weights average for conf_thresh = 0.25, TP = 91, FP = 0, FN = 0, average IoU = 88.67 % 
% yolo-obj_new_5000.weights average for conf_thresh = 0.25, TP = 91, FP = 0, FN = 0, average IoU = 91.63 %
% yolo-obj_new_6000.weights average for conf_thresh = 0.25, TP = 91, FP = 0, FN = 0, average IoU = 91.84 %
% yolo-obj_new_7000.weights average for conf_thresh = 0.25, TP = 91, FP = 0, FN = 0, average IoU = 92.06 %
% yolo-obj_new_8000.weights average for conf_thresh = 0.25, TP = 91, FP = 0, FN = 0, average IoU = 92.88 %
% yolo-obj_new_9000.weights average for conf_thresh = 0.25, TP = 91, FP = 0, FN = 0, average IoU = 92.86 %
% yolo-obj_new_10000.weights average for conf_thresh = 0.25, TP = 91, FP = 0, FN = 0, average IoU = 91.63 % 
% yolo-obj_new_11000.weights average for conf_thresh = 0.25, TP = 91, FP = 0, FN = 0, average IoU = 82.79 %
% yolo-obj_new_12000.weights average for conf_thresh = 0.25, TP = 91, FP = 0, FN = 0, average IoU = 94.10 % BESTA NETIÐ
% yolo-obj_new_13000.weights average for conf_thresh = 0.25, TP = 91, FP = 0, FN = 0, average IoU = 94.01 %
% yolo-obj_new_14000.weights average for conf_thresh = 0.25, TP = 91, FP = 0, FN = 0, average IoU = 88.15 %
%yolo-obj_new_15000.weights average for conf_thresh = 0.25, TP = 91, FP = 0, FN = 0, average IoU = 92.48 % 

% yolo-obj_new_last.weights average for conf_thresh = 0.25, TP = 91, FP = 0, FN = 0, average IoU = 93.19 %

