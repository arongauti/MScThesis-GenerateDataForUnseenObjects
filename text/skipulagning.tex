Lauslegir punktar til þess að hafa til hliðsjónar við skipulagningu ritgerðar.

Sjá einnig efni á neti um IMRaD (Introduction, Methods, Results, and Discussion) form fræðigreina.

Titill - Stuttur, hlutlaus og lýsandi 

Abstract

Introduction

Setja sviðið og afmarka viðfangsefnið.  Hvert er heildarverkefnið?  Hver er þinn þáttur?  Hver er tilgátan? Hvers vegna skiptir þetta máli?  
Staða þekkingar.  Hvað var búið að gera áður en þú byrjaðir?  Hvað var ekki búið að gera?
Bakgrunnur.  Hugsa út frá þörfum lesandans.  Hvað þarf hann að vita til þess að skilja það sem á eftir kemur?  Vitna í áður birt efni eða lýsa stuttlega eftir þörfum.
Ramma inn tilgátuna.  Hvaða spurningum á verkefnið að svara?  Hver er aðferðafræðin.

Design

Þessi kafli segir frá hönnunarforsendum og lausnum í verkefnum, sem innihalda umtalsverða hönnun.  Gerð grein fyrir uppbyggingu kerfis.  Annars sleppt.

Methods

Hér er sagt skipulega frá því hvernig tilraunir voru framkvæmdar, sem svara eiga spurningum verkefnisins og hvaða tól voru notuð til verksins.
Fara í hverja tilraun fyrir sig í sérstökum undirkafla.  Hvaða spurningu eða þætti spurningar á tilraun að svara?  Hvers vegna ætti hún að nægja til þess að svara spurningunni (tölfræði)?  
Lýsa uppsetningu og framkvæmd nægjanlega vel til þess að lesandinn geti endurtekið tilraunina.

Results

Hér eru niðurstöður tilrauna lagðar fram á hlutlægan hátt.  Upplýsingar eiga að nægja til þess að lesandi geti lagt sjálfstætt mat á niðurstöður (út frá eigin forsendum).
Hverjar voru niðurstöður hverrar tilraunar?  Töflur, gröf, dæmi og nægar upplýsingar til þess skilja gögnin sem lögð eru fram.
Vekja má athygli á því að einhver tala er stærri en önnur, en annars ekki lagt mat á niðurstöðurnar.

Discussion

Hér eru niðurstöðurnar túlkaðar og bornar saman við það sem vitað var fyrir.
Sagt frá takmörkunum og óvæntum niðurstöður.  Hverju bæta niðurstöður við stöðu þekkingar?

Conclusions

Tilgátan endurtekin.  Niðurstöður tilrauna teknar saman og dregnar af þeim ályktanir.  Hvað segja niðurstöðurnar um spurningarnar og tilgátuna, sem sett var fram í inngangi?
Hvað segja þær ekki?

Future work

Hvert yrði framhaldið, ef meiri tími væri til stefnu?  Hvaða spurningar aðrar vekja niðurstöðurnar?  Lagt mat á aðferðafræðina, sem beitt var.  Er ástæða til að reyna að nálgast viðfangsefnið á annan hátt?  Hugleiðingar, spin-off.


%\begin{table}[h]
% \resizebox{\textwidth}{!}{%
% \begin{tabular}{c|cccccccc}
% \hline
% \textit{Item} &
%   \textit{Products} &
%   \textit{Detections} &
%   \textit{True Positive} &
%   \textit{False Positive} &
%   \textit{Avg-IoU} &
%   \textit{Avg-Precision} &
%   \textit{Avg-Recall} &
%   \textit{Avg-F1} \\ \hline
% 1  & 372 & 106 & 106 & 0  & 0.8390 & 1.0000 & 0.3541 & 0.4824 \\
% 2  & 481 & 97  & 95  & 2  & 0.8399 & 0.9740 & 0.2622 & 0.3690 \\
% 3  & 99  & 36  & 35  & 1  & 0.8473 & 0.9688 & 0.4380 & 0.5705 \\
% 4  & 575 & 175 & 175 & 0  & 0.8599 & 1.0000 & 0.4021 & 0.5241 \\
% 5  & 458 & 87  & 52  & 35 & 0.5377 & 0.5592 & 0.1367 & 0.2057 \\
% 6  & 357 & 64  & 16  & 48 & 0.3148 & 0.2167 & 0.0559 & 0.0811 \\
% 7  & 378 & 73  & 58  & 15 & 0.7306 & 0.7803 & 0.1677 & 0.2646 \\
% 8  & 305 & 70  & 69  & 1  & 0.8495 & 0.9836 & 0.2781 & 0.4012 \\
% 9  & 255 & 48  & 43  & 5  & 0.7629 & 0.9167 & 0.2254 & 0.3225 \\
% 10 & 405 & 112 & 111 & 1  & 0.8468 & 0.9894 & 0.3449 & 0.4686 \\
% 11 & 535 & 185 & 185 & 0  & 0.8601 & 1.0000 & 0.4079 & 0.5420 \\
% 12 & 471 & 179 & 175 & 4  & 0.8352 & 0.9754 & 0.4523 & 0.5663 \\
% 13 & 349 & 104 & 100 & 4  & 0.7973 & 0.9588 & 0.3792 & 0.4952 \\
% 14 & 280 & 44  & 40  & 4  & 0.7625 & 0.9070 & 0.2448 & 0.3319 \\
% 15 & 299 & 76  & 71  & 5  & 0.7736 & 0.9338 & 0.2877 & 0.4118 \\ \hline
% \textit{\textbf{Average:}} &
%   \textit{\textbf{375}} &
%   \textit{\textbf{97}} &
%   \textit{\textbf{89}} &
%   \textit{\textbf{8}} &
%   \textit{\textbf{0.7638}} &
%   \textit{\textbf{0.8776}} &
%   \textit{\textbf{0.2958}} &
%   \textit{\textbf{0.4025}} \\ \hline
% \textit{\textbf{Total:}} &
%   \textit{\textbf{5619}} &
%   \textit{\textbf{1456}} &
%   \textit{\textbf{1331}} &
%   \textit{\textbf{125}} &
%   \textbf{-} &
%   \textbf{-} &
%   \textbf{-} &
%   \textbf{-} \\ \hline
% \end{tabular}%
% }
% \caption{The results when tested on unknown data, when zero detections are eliminated}
% \label{tab:v3zero}
% \end{table}

 %\begin{table}[h]
% \resizebox{\textwidth}{!}{%
% \begin{tabular}{c|cccccccc}
% \hline
% \textit{Item} &
%   \textit{Products} &
%   \textit{Detections} &
%   \textit{True Positive} &
%   \textit{False Positive} &
%   \textit{Avg-IoU} &
%   \textit{Avg-Precision} &
%   \textit{Avg-Recall} &
%   \textit{Avg-F1} \\ \hline
% 1  & 669  & 375 & 359 & 16  & 0.7256 & 0.9282 & 0.5964 & 0.6950  \\
% 2  & 1029 & 481 & 417 & 64  & 0.6008 & 0.7789 & 0.4641 & 0.5521 \\
% 3  & 667  & 413 & 385 & 28  & 0.7035 & 0.8810  & 0.6316 & 0.7135 \\
% 4  & 683  & 436 & 415 & 21  & 0.7272 & 0.9220  & 0.6599 & 0.7368 \\
% 5  & 840  & 262 & 175 & 87  & 0.4749 & 0.5766 & 0.2230  & 0.3082 \\
% 6  & 853  & 174 & 45  & 129 & 0.2638 & 0.1802 & 0.0604 & 0.0875 \\
% 7  & 851  & 381 & 297 & 84  & 0.5894 & 0.7090  & 0.3575 & 0.4622 \\
% 8  & 788  & 400 & 374 & 26  & 0.6932 & 0.9040  & 0.5070  & 0.6284 \\
% 9  & 880  & 369 & 299 & 70  & 0.5363 & 0.7131 & 0.3696 & 0.4707 \\
% 10 & 665  & 386 & 355 & 31  & 0.6802 & 0.8552 & 0.5609 & 0.6577 \\
% 11 & 627  & 432 & 407 & 25  & 0.7549 & 0.9159 & 0.6920  & 0.7679 \\
% 12 & 568  & 451 & 437 & 14  & 0.7542 & 0.9648 & 0.8019 & 0.8539 \\
% 13 & 615  & 345 & 335 & 10  & 0.7365 & 0.9473 & 0.5979 & 0.7027 \\
% 14 & 945  & 291 & 243 & 48  & 0.5918 & 0.7744 & 0.3091 & 0.4106 \\
% 15 & 602  & 333 & 315 & 18  & 0.7556 & 0.9196 & 0.5762 & 0.6771 \\ \hline
% \textit{\textbf{Average:}} &
%   \textit{\textbf{752}} &
%   \textit{\textbf{369}} &
%   \textit{\textbf{324}} &
%   \textit{\textbf{45}} &
%   \textit{\textbf{0.6392}} &
%   \textit{\textbf{0.7980}} &
%   \textit{\textbf{0.4938}} &
%   \textit{\textbf{0.5816}} \\ \hline
% \textit{\textbf{Total:}} &
%   \textit{\textbf{11282}} &
%   \textit{\textbf{5529}} &
%   \textit{\textbf{4858}} &
%   \textit{\textbf{671}} &
%   \textit{\textbf{-}} &
%   \textbf{-} &
%   \textbf{-} &
%   \textbf{-} \\ \hline
% \end{tabular}%
% }
% \caption{The results when tested on unknown data, when zero detections are eliminated}
% \label{tab:v2zero}
% \end{table}